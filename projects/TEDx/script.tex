%----------------------------------------------------------------------------------------
%	PACKAGES AND OTHER DOCUMENT CONFIGURATIONS
%----------------------------------------------------------------------------------------

\documentclass[a4paper, 11pt]{article} % Font size (can be 10pt, 11pt or 12pt) and paper size (remove a4paper for US letter paper)

\usepackage[protrusion=true,expansion=true]{microtype} % Better typography
\usepackage{graphicx} % Required for including pictures
\usepackage{wrapfig} % Allows in-line images

\usepackage{amsfonts} % Use the Palatino font
\usepackage[T1]{fontenc} % Required for accented characters
\linespread{1.05} % Change line spacing here, Palatino benefits from a slight increase by default

\makeatletter
\renewcommand\@biblabel[1]{\textbf{#1.}} % Change the square brackets for each bibliography item from '[1]' to '1.'
\renewcommand{\@listI}{\itemsep=0pt} % Reduce the space between items in the itemize and enumerate environments and the bibliography

\renewcommand{\maketitle}{ % Customize the title - do not edit title and author name here, see the TITLE block below
\begin{flushleft} % Right align
{\LARGE\@title} % Increase the font size of the title

\vspace{50pt} % Some vertical space between the title and author name

{\large\@author} % Author name
\\\@date % Date

\vspace{40pt} % Some vertical space between the author block and abstract
\end{flushleft}
}

%----------------------------------------------------------------------------------------
%	TITLE
%----------------------------------------------------------------------------------------

\title{\textbf{Banking 2.NO!}\\ % Title
Why Learning from the Past is Important for Our Future} % Subtitle

\author{\textsc{Owen Sims} % Author
\\{\textit{TEDx Cathedral Quarter}}} % Institution

%----------------------------------------------------------------------------------------

\begin{document}

\maketitle % Print the title section

%----------------------------------------------------------------------------------------
%	ABSTRACT AND KEYWORDS
%---------------------------------------------------------------------------------------

%----------------------------------------------------------------------------------------
%	ESSAY BODY
%----------------------------------------------------------------------------------------

\subsection*{Slide 1}

Tonight I want to talk about the future. Our future.\\

Unfortunately, for most, it's about the future of our economy and financial system. It can be heavy stuff. So, before I get into that, I first want to talk to you about a few parties that I've attended recently.\\

I don't get invited to many parties, but when I do I've noticed that there's always a point in the night when the mood changes, the conversation changes, it gets a bit philosophical... it gets a bit weird.\\

It's at this point when I know that I am in my element. This is the probably the reason I attended the party in the first place because at this point there is a question that I love to ask people. I love to ask it because I can get a pretty good indication of who they are and what they value in the world by their answer. And this is it...



\subsection*{Slide 2}

... "What do you think are mankind's greatest inventions?".\\

It's kind of a stupid question because there are no right answers. There are a host of potential candidates --- mankind has been pretty innovative --- but most people, who obviously haven't been thinking about this question, will reply with, "the wheel". Fair enough. Some people will give you different answers: medicine; the combustion engine; the Internet; the printing press; language; mathematics... they're all fair game.\\

However, one answer I never get is...


\subsection*{Slide 3}

... "Money".\\

In fact, when I bring it up people tend to scorn at me. Maybe because they feel that it's the root of all evil, or maybe because it's not a very good topic of conversation at a party. I'm not sure.\\

But money, or more generally methods of exchange, have been at the backbone of human development for millennia. We use them every day, multiple times a day, without even thinking about what money actually is, about the systems that underpin it, and why it all seems to work. We take it for granted.\\

That is, until things go wrong. Until we enter a crisis...


\subsection*{Slide 4}

... then we vilify it. We vilify the systems and we vilify the people that are meant to maintain it.\\

These times of Crisis happen more frequently than I originally thought and more frequently than we should probably allow...\\

It turns out that we still don't understand one of the very things that we depend on the most.


\subsection*{Slide 5}

We live in a world where new technologies --- such as Bitcoin and peer-to-peer lending services --- are being created to replace existing financial intermediaries.\\

They are being created to make a more stable and fair financial system, and are doing so with limited and questionable success.\\

But, could it be that our history is our greatest resource for creating a more stable and prosperous future? Can we learn anything from it?\\

It's a question worth asking. As Edmund Burke coined: "Those who do not learn from history are doomed to repeat it." It seems like he could be right.


\subsection*{Slide 6}

Apart from the Sub-prime Crisis --- which has left the legacy of millions unemployed, millions in poverty, and trillions of pounds in private and public debt --- the history of human civilisation is riddled with financial folly.


\subsection*{Slide 7}

In fact, every few years we seem to have another Panic --- and these are just a few.\\

These are not unexpected or random events: it's been shown that con-men, companies, organisations, financial institutions, and even Governments have blown and burst bubbles on purpose for their own benefit.\\

They share the same characteristics. They share the same causes. It just turns out that most economists and policy-makers are not so good at reading history... or indeed reading at all.


\subsection*{Slides 8 \& 9}

So, let's compare two tales of Crisis: the first in 1857 and the next 150 years later in 2007.



\subsection*{Slide 10}

By 1857 the world was already well versed in Crises.\\

Both America and the UK seemed to be operating on a 'one-crisis-per-decade' rule:
\begin{itemize}
\item In 1825 both suffered from Crises involving investments into Latin American Wars and fictional Latin American countries;
\item In 1837 we had a crises in cotton trade in the UK; 
\item In 1847 we had a crisis regarding a collapse of railway stock.
\end{itemize}

So by 1857 when Rail-road stock crashed again and bankers began running their own banks, people thought that this was more of the same.\\

However, since the last Crisis the world had changed... By then Britain had strengthened economic and social relationships with the US and the rest of Europe. Low interest rates and the establishment of new forms of joint-stock banks and discount houses had increased deposits and debts throughout the UK and Europe.



\subsection*{Slide 11}

So when Panic transmitted across financial institutions in America, the effects were felt throughout the rest of the West.\\

The perils of a HyperConnected World, which we talk about today, were noted in 1857 as a contagion of panic spread throughout developed economies.\\

This was the first ever global crisis and as the Economist newspaper noted at the time: it was the "\textit{greatest crisis in history}".


\subsection*{Slide 12}

The financial and economic networks had become so intertwined that any spark in the network could cascade through the coupled nodes and cause problems in the entire system.\\

All networks, whether they be economic, financial, social, biological, or neural, share the same characteristics. One of which is a so--called \textit{small--world}: any two nodes in the network can be connected through an extremely small chain. A contagious process does not have to go far before the entire network is impacted.


\subsection*{Slide 13}

The characteristics of the 1857 Crisis are compelling...

\begin{itemize}
\item[(1)] Global panic was triggered by a series of bank-runs;
\item[(2)] There was an extended period of low interest rates to stimulate economic activity after the previous crises. This encouraged borrowing and growth based on debt;
\item[(3)] There was a host of financial innovations, notably the discount house, created to increase deposits and increase lending;
\item[(4)] Rail-roads were the dot-com's of their day. New technologies to speculate over;
\item[(5)] There was no real regulation of risky ventures with depositors money.
\end{itemize}




\subsubsection*{Slide 14}

Fast-forwarding 150 years into the future and the characteristics are the same.

\begin{itemize}
\item[(1)] The Crisis began with a bank-run. Northern Rock in 2007;
\item[(2)] There had been persistently low interest rates, making mortgages cheaper;
\item[(3)] Exotic financial innovations had been created such as CDO's, CDS's, MBS's, and other ABS's;
\item[(4)] A host of new technologies had been created -- the returns to which are relatively unknown. Again, these are perfect to speculate over;
\item[(5)] Again, there was no real regulation of risky ventures with depositors money.
\end{itemize}


\subsubsection*{Slide 15}

Not only are the characteristics the same, but so too are the causes.\\

In fact, there is one cause consistent in all crises: shareholders, the actual owners of financial institutions, have no incentive to be prudent.\\

They have no incentive to lend your deposits in a safe manner. Regardless of what happens, the system is structured in such a way that the expected gain from consistently undertaking risky ventures is greater than any expected loss from the financial institution failing\\

Read quote from John to back this claim...


\subsubsection*{Slide 16}

We need incentives, we need regulation.\\

Political philosopher Karl Marx claimed that history repeats itself, ``\textit{first as a tragedy, then as a farce.}'' We are far past the `farce' stage of Crisis. It is time to stop relying so much on new financial technologies, and start learning from the problems of our own history.

\end{document}